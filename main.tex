\documentclass[12pt]{article}
\usepackage{graphicx}
\usepackage{caption}
\usepackage{amsmath}
\usepackage{geometry}
\usepackage{float}
\usepackage{booktabs}
\usepackage{longtable}
\usepackage{setspace} % for spacing control
\usepackage{indentfirst}
\usepackage{titlesec}
\setstretch{1.0}       % single line spacing (default)
\setlength{\parskip}{0pt}   % no space between paragraphs
\setlength{\parindent}{0em} % set indentation amount

\geometry{margin=1in}
\setlength{\parskip}{0em}
\setlength{\parindent}{0pt}

\title{Assignment 1 \\ \large Computational Plasticity (SoSe25)}
\author{Bagus Alifah Hasyim \\ 108023246468 \\ Last Three Digits: 468}
\date{}

\begin{document}
\maketitle

\section*{Given Data}
\hspace*{2em}In this section we need to define every data that is given in the assignment, such as the material properties, geometry, 
and other relevant parameters that are needed for the analysis. Therefore, following data shall we define using the author's "immatrikulation nummer":

\begin{itemize}
    \item $E_a = 200 + (10 \times 4) = 240 \;\text{GPa}$  
    \item $\sigma_a = 300 + (10 \times 6) = 360 \;\text{MPa}$
    \item $\sigma_{yb} = 200 + (10 \times 8) = 280 \;\text{MPa}$
\end{itemize}

\section{Introduction}
\hspace*{2em}In this assignment, we will analyze the mechanical response of a baseline plate
and a plate with a circular inclusion under uniaxial loading. 
The analysis will include comparison of obtaining stress-strain curves between the force-displacement and
direct stress-strain curves, differentiating plane stress and plane strain conditions, comparing 
local stress field distribution based on different constitutive models, and evaluating stress at specific points based on mesh sizes.

\hspace*{2em}For the following sections we will systematically address each question in the assignment, 
providing detailed explanations, derivations, and relevant figures or tables to 
support the analysis. 
All calculations and results are based on the provided data and the author's unique 
identification number. Numerical analysis will be performed using Abaqus CAE with an appropriate
boundary condition and settings. 

\section{Set up of the Model}
\hspace*{2em}For the model setup, we will use Abaqus CAE to create a 2D planar-deformable shell model of the plate with the specified geometry, boundary condition, material properties, and mesh.  
\subsection{Geometry and Boundary Conditions}
%add axis in the figure, for direction 1, 2, and 3
\begin{enumerate}
     \item \textbf{Baseline Plate:} A rectangular plate with length $L = 40 \;\text{mm}$, width $W = 10 \;\text{mm}$, and thickness $t = 1 \;\text{mm}$.
        \begin{figure}[H]
        \centering
            \includegraphics[width=0.5\textwidth]{images/TaskQ1.png}
        \caption{Baseline plate geometry with defined material A. Two boundary conditions are applied, which are the 
        fixed support restriction in y direction on the bottom line of the plate, fixed support point on the bottom right point,
        and constant distribution of displacement in y direction on the top line of the plate, which has value of 5 mm.}
        \label{fig:geometryQ1}
\end{figure}

    \item \textbf{Plate with Circular Inclusion:} Same as the baseline plate, but with a circular inclusion of radius $r$ located at the center or specified position.
\begin{figure}[H]
        \centering
            \includegraphics[width=0.52\textwidth]{images/TaskQ2.png}
        \caption{Plate with circular inclusion and defined material B. Boundary conditions are the same as in Task 1 shown in Figure~\ref{fig:geometryQ1}.}
        \label{fig:geometryQ2}
\end{figure}
\end{enumerate}
\subsection{Material Properties}
\hspace*{2em}Material properties for the analysis are provided from the task definition, which there are some
material parameters that need to be manually calculated based on the author's Immatrikulation Nummer in at very 
first section. 
For task 1, material A (see Table~\ref{tab:materialA-properties}) will be used and defined as an elastic-plastic material with properties based on the Young's modulus $E$, Poisson's ratio $\nu$, and yield stress $\sigma_{ya}$,
and ultimate tensile strength $\sigma_{f}$, and the strain at fracture $\epsilon_{f}^{f}$ in isotropic condition.   
For task 2 with additional inclusion material B in the center, it is defined as an elastic-plastic material with anisotropic properties, 
which it defines the properties differently in three orthogonal directions. Especially for the plastic properties in table~\ref{tab:materialB-plasticproperties},
it used Hill's yield criterion, which is defined by the Hill's coefficients $R_{ij}$, where $i,j = 1,2,3$ for the three orthogonal directions. 


\begin{table}[H]
    \centering
    \caption{Elastic and plastic properties of material A.}
    \label{tab:materialA-properties}
    \begin{tabular}{lllll}
        \toprule
        E (GPa) & $\nu$ & $\sigma_{ya}$ (MPa) & $\sigma_{f}$ (MPa) & $\epsilon_{f}^{f}$ \\
        \midrule
        240 & 0.3 & 360 & 550 & 0.4 \\
        \bottomrule
    \end{tabular}
\end{table}

\begin{table}[H]
    \centering
    \caption{Elastic properties of material B.}
    \label{tab:materialB-elasticproperties}
    \begin{tabular}{lllllllll}
        \toprule
            \centering $E_1$ (GPa) & $E_2$ (GPa) & $E_3$ (GPa) & $\nu_{12}$ & $\nu_{13}$ & $\nu_{23}$ & $G_{12}$ (MPa) & 
            $G_{13}$ (MPa) & $G_{23}$ (MPa) \\
            \midrule
            \centering 210 & 220 & 230 & 0.3 & 0.31 & 0.32 & 80 & 84 & 87 \\
            \bottomrule
    \end{tabular}
\end{table}

\begin{table}[H]
    \centering
    \caption{Plastic properties of material B.}
    \label{tab:materialB-plasticproperties}
    \begin{tabular}{lllllllll}
        \toprule
            \centering $R_{11}$ & $R_{22}$ & $R_{33}$ & $R_{12}$ & $R_{13}$ & $R_{23}$ & $\sigma_{yb}$ & 
            $\sigma_{f}$ (MPa) & $\epsilon_{p}^{f}$  \\
            \midrule
            \centering 1 & 1.2 & 1.25 & 0.8 & 0.85 & 0.95 & 280 & 450 & 0.5 \\
            \bottomrule
    \end{tabular}
\end{table}
\subsection{Meshing}
\section*{Results and Discussion}

\subsection*{Q1: Mechanical Response Analysis for Baseline Plate}
\begin{itemize}
    \item[(a)] XXX…
    \item[(b)] XXX…
    \item[(c)] XXX…
\end{itemize}

\subsection*{Q2: Mechanical Response Analysis for Plate with Circular Inclusion}
XXX…

\subsection*{Q3: Local Stress Field Distribution and Analysis}
XXX…

\subsection*{Q4: Comparison Analysis at Point A}
XXX…

\subsection*{Q5: Comparison Analysis at Point B}
XXX…

\subsection*{Extra Task}
To calculate $\frac{\partial{f}}{\partial{\sigma}}$ with Voigt notation tensor $\sigma = (\sigma_1, \sigma_2, \sigma_3, \sigma_4, \sigma_5, \sigma_6)$ from yield function 
$f(\sigma)=\sigma_{eq} - \sigma_y$, where $\sigma_{eq}$ can be described as follows:

\begin{equation}
\sigma_{eq} = \sqrt{\frac{1}{2} \left( (\sigma_1 - \sigma_2)^2 + (\sigma_2 - \sigma_3)^2 + (\sigma_3 - \sigma_1)^2 + 6(\sigma_4^2 + \sigma_5^2 + \sigma_6^2) \right)}
\end{equation}

Hence, we can simplify the prescribed equation to:

\begin{equation}
A = \frac{1}{2} \left( (\sigma_1 - \sigma_2)^2 + (\sigma_2 - \sigma_3)^2 + (\sigma_3 - \sigma_1)^2 + 6(\sigma_4^2 + \sigma_5^2 + \sigma_6^2) \right)
\end{equation}

We can substitute the equation above into the yield function $f(\sigma)$ and applying the chain rule in the 
derivative, we obtain:

\begin{equation}
f(\sigma) = \sqrt{A} - \sigma_y \rightarrow \frac{\partial{f}}{\partial{\sigma}} = \frac{\partial{f}}{\partial{A}} \cdot \frac{\partial{A}}{\partial{\sigma}} = \frac{1}{2\sqrt{A}} \cdot \frac{\partial{A}}{\partial{\sigma}}
\end{equation}

Since we defined $A$ as a substituent of the equivalent stress $\sigma_{eq}$, we can revert the equation as in the origin form:

\begin{equation}
    \frac{\partial{f}}{\partial{\sigma}} = \frac{1}{\sigma_{eq}} \cdot \frac{\partial{A}}{\partial{\sigma}}
\end{equation}

Now we need to expand the tensor differential with respect to each stress component in 
the Voigt notation. Therefore, we can do the differential operation for each stress 
component and assemble them to a Voigt notation as following sequence:

\[
\frac{\partial{A}}{\partial{\sigma_1}} = H_{1}(\sigma_1-\sigma_2) + H_3(\sigma_1-\sigma_3) \;;
\]
\[
\frac{\partial{A}}{\partial{\sigma_2}} = H_{2}(\sigma_2-\sigma_3) + H_1(\sigma_2-\sigma_1) \;;
\]
\[
\frac{\partial{A}}{\partial{\sigma_3}} = H_{2}(\sigma_3-\sigma_2) + H_3(\sigma_3-\sigma_1) \;;
\]
\[
\frac{\partial{A}}{\partial{\sigma_4}} = 6H_{4}\sigma_4 \;;
\]
\[
\frac{\partial{A}}{\partial{\sigma_5}} = 6H_{5}\sigma_5 \;;
\]
\[
\frac{\partial{A}}{\partial{\sigma_6}} = 6H_{6}\sigma_6 \;;
\]

To summarize the above equations, we can write them in a matrix form as follows:
\begin{equation}
\frac{\partial{f}}{\partial{\sigma}} = \frac{1}{\sigma_{eq}}\begin{bmatrix}
H_{1}(\sigma_1-\sigma_2) + H_3(\sigma_1-\sigma_3) \\    
H_{2}(\sigma_2-\sigma_3) + H_1(\sigma_2-\sigma_1) \\
H_{2}(\sigma_3-\sigma_2) + H_3(\sigma_3-\sigma_1) \\
6H_{4}\sigma_4 \\
6H_{5}\sigma_5 \\
6H_{6}\sigma_6
\end{bmatrix}
\end{equation}




\section*{Conclusion}
(Summarize your findings and state key takeaways from the analysis)



\end{document}

